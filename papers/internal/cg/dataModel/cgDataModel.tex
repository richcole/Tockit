\documentclass{article}

\usepackage{hyperref}
\usepackage{thumbpdf}

\newcommand{\defname}{\textit}
\newcommand{\comment}{\textbf{Comment: }}

\title{A Simplified Data Model for CGs}

\author{Peter Becker and Joachim Hereth}

\pagestyle{plain}

\begin{document}

\maketitle

\begin{abstract}
This paper tries to model a simplified data model for CGs that can be
used as base for different notations and applications. Based on the
draft for an ISO standard the proposed model splits CGs into a core
with only a simple structure and gives a number of extensions to
introduce additional features in a consistent way.
\end{abstract}

\section{Introduction}

This paper proposes a data model for simple Conceptual Graphs (positive
Conceptual Graphs with untyped nesting and without generic markers,
according to the classification by \cite{CM97}). It is
based upon the current\footnote{Date of writing: September 01, 2001} draft for
an ISO standard for Conceptual Graphs edited by John Sowa \cite{standard}
and discussions with various people involved in CGs.

The main purpose of this approach is to give a structure for treating
Conceptual Graphs in programs and notations. We distinguish between
three aspects for modelling Conceptual Graphs: a data model, a set of
operations and the syntactical notations. In this paper we will use
the term ``syntax'' only for specific notations, while the term
``semantic'' refers to the algebra defined by the data model and the
operations upon it. This view is taken since we consider CGs themself
as the domain we want to model, as opposed to many other papers which
consider CGs as syntax and try to model their semantics.

The data model described in section~\ref{sec:dataModel} models only
the parts needed for simple CGs, many aspects described in the
standard draft are deliberately skipped since they are considered as
extensions and this model should only be a base to work with.

In section~\ref{sec:extensions} we discuss possible extensions of the
data model given, some of them can be used to get the expressibility
of the current Conceptual Graphs, some of them go beyond the usual
Conceptual Graphs. Section~\ref{sec:summary} gives a summary and
discussion of the paper.

\section{Data Model}
\label{sec:dataModel}

A data model is the core of every application in an IT system. Often this
important aspect is not modelled explicitely and discussions showed that the
CG community has no coherent definition of this data model. The data model
given in this section is meant to build a referable base model for CGs, at
least for purposes of discussion.

The goal of a data model is to be well-defined but
understandable. Although the data modelled here is usually seen in the
syntactical form of the CG Display Form (DF) it is not meant to model
this syntax but the data represented by it. In many cases this will be
more complex than the usual representations in DF but if a simpler
representation is possible for some data a specific syntax can always
allow shortcuts like declaring parts of the alphabet implicitly. This
kind of information has still to be modelled for purposes of
implementation in IT systems and reasoning.

\subsection{Catalog}

The \defname{catalog} defines in this data model the basic tokens used
in the graphs themself. The term ``catalog'' is used here not only for
the tokens themself but also for the additional structures defined on
them, i.e. the type hierarchies.

\comment{The idea of modelling this as CGs themself is surely
interesting but we decided to skip it due to some problems in this
approach. Since the data model doesn't describe how exactly the
information is stored or written an implementation or notation can
still use this approach.}

\subsubsection{Type}

The \defname{types} are a set of basic objects on which a partial order is
defined by a \defname{subtype} relation which has a greatest element
called \defname{universal type} denoted with $\top$.

\comment{We dropped the bottom element since it does not seem important
for either implementations or notations.}

\subsubsection{Relation}

The \defname{relations} are a set of basic objects. Each relation has
 a \defname{signature}. The signature of a relation is defined by a vector
 of $n$ types, $n$ is called the \defname{valence} of the relation.

For each valence there exists a partial order defined by a
\defname{subtype} relation with a greatest element $\top_n$ that has
the signature with a list of $n$ occurences of $\top$. If a relation is a
subtype of another its signature has to be a subtype of the other in
each element.

\comment{This is what describes mathematically the relations themself,
not a type of relation, while the vertices in a graph only model one
element in the relation. Or to be more exact: the relation is modelled
intensionally by this structure and extensionally by all links which
belong to it (see Sec.~\ref{sec:links}).

To ensure well-defined subtyping we introduced multiple top elements for
relations, one for each valence. This way we can model subtypes of relations
while ensuring correct signatures.}

\subsubsection{Instances}

The third kind of core data are the \defname{instances}. These don't have
any hierarchies, they are just a simple set having unique markers as indices.

\comment{In this model all instances have to be refered to by their markers,
there is no way to talk about anything specific without declaring an
instance, although this can be done implicitly in a syntax.}

\subsection{Conceptual Graph}

A \defname{conceptual graph} is modelled as bipartite graph from
\defname{nodes} and \defname{links} where the links are explicitely modelled
as vertices in the data model. The arcs of such graph are
modelled through references from a link to a node.

\subsubsection{Node}

A \defname{node} has a \defname{type}, a \defname{referent}
pointing to an instance and it can have a conceptual graph used as
\defname{descriptor}.

If a descriptor is given the graph and all nodes and links directly in
it are called to be \defname{immediately nested} in the node
containing the descriptor. Any node or link $c$ is called
\defname{nested} in another node $d$ if it is immediately nested in
$d$ or nested in some node immediately nested in $d$.

Two nodes or links $c_1$ and $c_2$ are said to be \defname{co-nested}
if they are either the same or immediately nested in the same node
$d$. Any node or link $c$ nested in another node $d$ is said to be
\defname{more deeply nested} than any node or link co-nested with $d$.

A node or link $d$ is said to be \defname{within the scope} of a node
or link $c$ if it is co-nested with $c$ or more deeply nested
than $c$.

\comment{This has been highly simplified, excluding a number of features
we considered only syntactical aspects and some others that can be modelled
as extentions (see section~\ref{sec:extensions}).

Some of the main differences are: only the existential quantifier is
allowed (the universal is available when negation is added,
collections are considered as extension), no literals (might be
replaced by representations), only one locator (the other ones don't
map into the data model).

The term ``node'' has been choosen since it matches the typical usage in
semantic networks and its interpretation in a CG depends on the quantifier
used -- used with the existential quantifier it refers more to a specific
individual than a concept.}

\subsubsection{Link}
\label{sec:links}

A \defname{link} has a relation as its \defname{type} and has a
vector of $n$ \defname{references} to nodes where $n$ is
the valence defined for its relation. Each node refered has to be
either of the same type or a subtype of the type given in the same
position in the signature for its relation and the link has to be in
the scope of the node.

\comment{The ability to reference nodes outside the scope of the link
and not only those co-nested with it removes the need for coreference
sets.

To map this model into coreference sets one could replace the
reference to a node with a node of type $\top$ without referent and a
coreference link to the node refered.}

\section{Data Model Extensions}
\label{sec:extensions}

This section describes a number of extensions that can be made to make
the basic data model more expressive. Some of them are already covered
at least partly in \cite{standard}, some are already implemented in
tools and others are just ideas that were raised in discussions.

\subsection{Representations}

For any instance a number of mappings into a set of
\defname{representations} can be given. These representations are some
kind of encodings (e.g. Unicode strings as names, numbers, binary
data) that can be processed by specific applications.

To ensure interoperability between different applications a common set
of encodings should be defined and there should be some way to extend
this set in a way that any application can decide if it can present a
specific encoding or not.

One option for defining a core set of encodings would be reusing the
datatypes defined for XML Schema (see
\cite{XMLS-DT}). Additional encodings could be defined by
reusing the MIME-types (\cite{MIME}).

\subsection{Generic Marker}

In addition to the individuals a \defname{generic marker} can be
allowed which denotes the existence of some individual of the given
type without defining the exact one.

\subsection{Negation}
\label{sec:negation}

Negation can be done by introducing a new construct for negating a
graph. In the description of CGIF this is describe as a ``concept of
type Proposition with an attached relation of type Neg'' but in our
view it is a completely new construct and should be modelled as such.

This means introducing a construct \defname{negated graph} which can
occur in any conceptual graph and is a conceptual graph
itself. References across the border of this graph are allowed as for
the descriptors of nodes, this way cuts can be modelled as presented
in \cite{Da00} which is equivalent to the way John Sowa proposes the
modelling using co-references, a display form which can be achieved by
showing additional nodes with type $\top$.

\subsection{Collections}

To allow a smaller data model for repetitive graphs the referent of a
node could be extended to contain a set of instances instead of a
single one which can be read as having a number of graphs where each
one establishes the same link for each of the individuals in the
set.

\subsection{Additional Quantifiers}

When negation has been introduced the universal quantifier can be
modelled easily using the existential one in combination with
negation.  In addition to these options one might be interested in
modelling collection sizes explicitely, e.g. to state something like
``there are at least three different people wearing the same type of
shirt''. As opposed to the collections described above the exact
instances don't have to be known. Minimum and maximum size could be
given. This would map into simple graphs in a similar way as the
defined collections where a number of implicit distinct instances
will be created.

\subsection{Lambda Expressions}
\label{sec:lambdaExpressions}

Lambda expressions can be modelled by adding \defname{formal parameters}
into the conceptual graphs. They could be interpreted in the same manner
as formal parameters in a declaration in a programming language, defining
a signature with a number and the type of concept instances accepted to
instantiate this lambda expression.

Additionally there has to be some way to add instances of lambda
expressions into a conceptual graph giving the \defname{actual
parameters} needed for instantiation. The types and relations have
to allow specifying monadic lambda expressions instead of basic objects.

Since lambda expressions are just graphs with formal parameters we
don't think they should be modelled as a separate construct since there
doesn't seem to be a need for it. Every graph can be read as $n$-adic
lambda expression where $n$ is the number of formal parameters given
for it.

\subsection{Cross-Document Linking}

For many purposes it would be useful to be able to link documents
together in different ways. For example, one could use an alphabet
defined in a central file to discuss graphs given elsewhere with it.

To accomplish this one has to introduce the notion of a document and to give
some way of giving cross-document references in the graphs and for the
catalog parts. In some way this can be seen as specialisation of the
extisting structures that accomodate with the cross-document
referencing issues -- nothing new will be introduces but the existing
model has to be more specific to be able to handle e.g. concept
references in a consistent way across documents.

One of the most central issues in this is having globally unique
identifiers.  While the current model in a form for one document can
use e.g. simple numbers for identifying types, relations and
individuals this is not sufficient to ensure that the modelled
information can be refered to by an external document.

These problems already have been addressed in the XML field, a
technique like XLink (\cite{XLink}) can be used for establishing the
links while the concept of namespaces (\cite{XML-NS}) can be used
to avoid conflicts.

\subsection{Actors}

Another extension to introduce new expressivity is adding something
for implementing functionality like calculations (\cite{So98}). This
is done using actors but there is no common model for this
yet. In addition to the actors themselves this needs some way to model
variables that can be changed by them.

This extension is rather complex and will not be discussed here.

\subsection{Rules}

While actors change variables in a graph many applications might want
to change the graphs themself. This can be done by using colored or
marked graphs like those presented by Baget (\cite{Ba00}) or implemented in
the Ossa system. This is another extension not discussed here.

\subsection{Reifying types and relations}

The model presented here does not allow reasoning on the types or the
relation themself. This might be useful sometimes but needs some kind
of extension, too. This could be done using a specific type of nodes
in a conceptual graph for them.

Another part of CGs that can't be used in CGs with this model are the
links themself. If one drops the idea of the bipartite graph and
allows links to refer to links this could be achieved, too. The
hierarchies could get a common top element to model this more easily
(a supertype of $\top$ and all $\top_n$ nodes).

\section{Operations}

Given this data model operations on it can be defined to give an
algebra for using CGs. This algebra will define all syntactical
operations that can be done on CGs, the term ``syntactical'' refering
here to CGs as syntax for knowledge representation.

There has some work be done to model these operations in a formal
mathematical way (\cite{Pr00}, \cite{Mi00}), we expect that this work
can be easily reused for defining the operations on the given data
model.  The data model presented here gives more detail and is closer
to implementations and notations but should have a direct mapping to
the mathematical structures used in the existing work.

\section{Summary}
\label{sec:summary}

The data model presented in this paper can be used as a base model for
developing CG applications and notations. Since this model is
deliberately kept simple and left many of the historic burdens of
existing formats and applications aside it should be a good base for
discussions and further work. One of the most important approaches is
to add an algebra to it and to show that thereby one gets a semantic
that is useful for CGs. In addition at least some of the extensions
should be modelled more explicitely.

The authors hope to give useful input with this into the process of
creating a standard for CGs and wish to thank everyone who discussed
the data model with us, most noticeably Finnegan Southey and Murray
Altheim.

\begin{thebibliography}{ABCDE99}

\bibitem[Ba00]{Ba00} J.-F.\,Baget: Extending the CG Model by
  Simulations, in: B.\,Ganter, G.W.\,Mineau (Eds.):
  \textit{Conceptual Structures: Logical, Linguistic, and Computational
  Issues}, LNAI 1867, Springer-Verlag, Berlin Heidelberg 2000,
  pp. 277--291

\bibitem[CM97]{CM97} M.\,Chein, M.-L.\,Mugnier: Positive Nested
  Conceptual Graphs, in: D.\,Lukose et al. (Eds.): \textit{Conceptual
  Structures: Fulfilling Peirce's Dream}, LNAI 1257, Springer Verlag,
  Berlin-New York 1997, pp. 95--109.

\bibitem[Da00]{Da00} F.\,Dau: Negations in Simple Concept Graphs, in:
  B.\,Ganter, G.W.\,Mineau (Eds.):
  \textit{Conceptual Structures: Logical, Linguistic, and Computational
  Issues}, LNAI 1867, Springer-Verlag, Berlin Heidelberg 2000, pp. 263--276

\bibitem[Mi00]{Mi00} G.\,Mineau: The Extensional Semantics of the Conceptual
  Graph Formalism, in: B.\,Ganter, G.W.\,Mineau (Eds.):
  \textit{Conceptual Structures: Logical, Linguistic, and Computational
  Issues}, LNAI 1867, Springer-Verlag, Berlin Heidelberg 2000, pp. 221--234

\bibitem[MIME]{MIME} E.\,Hood: \textit{Multipurpose Internet Mail Extensions
  (MIME)}, Summary Website,\\
  \href{http://www.nacs.uci.edu/indiv/ehood/MIME/MIME.html}
       {http://www.nacs.uci.edu/indiv/ehood/MIME/MIME.html}

\bibitem[Pr00]{Pr00} S.\,Prediger: Nested Concept Graphs and Triadic
  Power Context Families, in: B.\,Ganter, G.W.\,Mineau (eds.):
  \textit{Conceptual Structures: Logical, Linguistic, and Computational
  Issues}, LNAI 1867, Springer-Verlag, Berlin Heidelberg, pp. 249--262.

\bibitem[So98]{So98} J.\,Sowa: Conceptual Graph Extensions, in: M.-L.
  Mugnier, M.\,Chein (Eds.): \textit{Conceptual Structures: Theory,
  Tools and Applications}, LNAI 1453, Springer-Verlag, Berlin Heidelberg,
  pp. 3--14

\bibitem[So01]{standard} J.\,Sowa: \textit{Conceptual Graphs}, Draft for
  an ISO International Standard,
  \href{http://www.bestweb.net/~sowa/cg/cgstand.htm}
       {http://www.bestweb.net/~sowa/cg/cgstand.htm}

\bibitem[XLink]{XLink} W3C Recommendation: \textit{XML Linking Language
  (XLink), Version 1.0},
  \href{http://www.w3.org/TR/2001/REC-xlink-20010627}
       {http://www.w3.org/TR/2001/REC-xlink-20010627}

\bibitem[XMLS-DT]{XMLS-DT} W3C Recommendation: \textit{XML Schema Part 2:
  Datatypes, Version 1.0},
  \href{http://www.w3.org/TR/2001/REC-xmlschema-2-20010502}
       {http://www.w3.org/TR/2001/REC-xmlschema-2-20010502}

\bibitem[XML-NS]{XML-NS} W3C Recommendation: \textit{Namespaces in XML,
  Version 1.0},
  \href{http://www.w3.org/TR/1999/REC-xml-names-19990114}
       {http://www.w3.org/TR/1999/REC-xml-names-19990114}

\end{thebibliography}

\end{document}













